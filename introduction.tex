%!TEX root = ./FMi_2015_AISC.tex
%
% UTF8-Check: Umlaute: äöüÄÖÜß
%
\label{sec:introduction}
%
\emph{Scientific workflows} are a special class of workflows, which are characterized as large-scale, long-running and resource-intensive  \cite{scientific}. 
%Different domains can benefit from the Cloud. 
%
The major goal of scientific workflows is to allow scientists to focus on domain-specific (science) aspects of their work, rather than dealing with complex data management and software issues \cite{scientific20}. 
%In the current work, we focus on \emph{scientific workflows}, which are characterized as large-scale and long-running.
%
How the sequence of the workflow tasks is represented can be handled in different ways.
%
The sequence can be specified either by scripting languages or through graph-based techniques such as Petri nets or $ \pi $-calculus \cite{Aalst03picalculus} \cite{smith04pi}.
%
Scripting languages are usually based on markup languages such as
Extensible Markup Language (XML). They may be convenient for well skilled users and do not need to be converted to be run on a cloud environment. But, they are not user-oriented and do not permit to  specify large and complex workflows manually \cite{hayatndt}. 
%
In this work, Petri nets \cite{petri21} and more specifically \textit{reference nets} \cite{Kummer02} are presented as a modeling technique suitable to model the workflow patterns
in an elegant and easy way \cite{sofiane}.

Nevertheless, due to the large amount of data and tasks, that need to be processed, the execution of such kind of workflows requires often to be mapped into external resources.
%
During last few years, Cloud computing is growing considerably and it is gaining popularity in both academia and industry.
%
We believe that Clouds are the suitable environement for the execution of scientific workflows thanks to the powerful resources, which range from storage, computing and networks that clouds provide.
%
Unfortunately, existing Workflow Management Systems are not adapted to perform in the Cloud.
%
They need to fit to the Cloud architecture.
%
They also need to provide modeling means and migration mechanisms that enable a full integration between workflow concepts and Cloud technology.


The objective of the current work is twofold.
%
First, we implemented a tool named \RenewGrass{} for the specification and the execution of scientific workflows.
%
Then, we, successfully, integrated \RenewGrass{} in the \textbf{RE}ference \textbf{NE}ts \textbf{W}orkshop (\Renew{}) which is available at (\url{www.renew.de}), our chosen modeling and simulation tool for Petri nets.
%
%\RenewGrass{} has been successfully integrated in the \textbf{RE}ference \textbf{NE}ts \textbf{W}orkshop (\Renew{}) which is available at (\url{www.renew.de}), our chosen modeling and simulation tool for Petri nets.
%
As a domain of application, the tool is suitable for remote sensing especially processing of satellite imagery.
%
% \cite{Diverse Papiere zu RenewGrass!}
%
Modeling and implementing such kind of workflows need specific tools and techniques, which are unfortunately not provided by \Renew{}.
%
Technically, the integration of \RenewGrass{} consists on extending  \Renew{} by modules and components of the Geographic Resource Analysis Support System (\textsc{Grass}) GIS (Geographic Information System).
%Technically, the Geographic Resource Analysis Support System (\textsc{Grass}) GIS (Geographic Information System) modules \cite{GRASS_GIS_software} have been successfully integrated with \Renew{}. 
%
This allows users to invoke Grass GIS modules directly from their Petri net models, which can be later executed.
%
With the integration of \RenewGrass{} into \Renew{}, the latter is now able to deal with other research domains such as the scientific domain.
%
Moreover, as soon as the workflow requirements become locally unsatisfied, workflow tasks need to be mapped to distributed resources.
%
%With respect to the work presented in \cite{Bendoukha+15c} and as an extension, this issue is also taken into account, since our long-term perspective is to provide a service-oriented environment built on top of Cloud resources and to allow flexible deployment of scientific workflows.
%
Therefore, we will discuss the deployment of \RenewGrass{} into the Cloud.
%
Questions like: Where to store the data? Where to execute the activities? will be investigated.
%
Here, we propose an agent-based architecture, where each functionality of the system is performed by a specific agent.
%
In our approach, we follow the \Paosel{} (\Paose{}) paradigm for developing agent-based applications.
%

%%Remainder of the paper
The rest of the paper is organized as follows. 
%
Related work as well as the conceptual and technical background of this work are presented in Section.~\ref{sec:functionality}.
%
\RenewGrass{} is described in Section.~\ref{sec:grassintegration}. 
%
How \RenewGrass{} can be deployed in the Cloud is investigated in Section.~\ref{sec:Cloudmigration}.
%
%The drawbacks of the current implementations and propose further solutions are discussed in Section.~\ref{sec:discussion}.
Section.~\ref{sec:discussion} discusses further solutions.
%
Section.~\ref{sec:conclusion} concludes the paper with summary and future works.





